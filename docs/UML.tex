\documentclass[margin=10pt]{standalone}

\usepackage{tikz}
\usepackage{amsmath}
\usepackage{mathtools}
\usepackage{microtype}
\usepackage[dutch]{babel}

\usepackage{inconsolata}
%% computer modern monospace?
%% \usepackage[tt={oldstyle=false,variable=false}]{cfr-lm}

\usepackage[T1]{fontenc}

\usepackage{listings}
\usepackage{minted}
\usemintedstyle{vs}

%% Color definitions
\usepackage[dvipsnames]{xcolor}
\definecolor{githubgrey}{HTML}{EAEEF2}
\definecolor{darkgreen}{HTML}{3D5E45}
\definecolor{purple}{HTML}{5C5573}
%% rounded boxes
\usepackage[most]{tcolorbox}

%% for templates
\usepackage{xparse}
\newcommand\mystrut{\rule[-2.5pt]{0pt}{12pt}}
%% datatype typesetting
\newcommand{\datatype}[2][text]{%
    \tcbox[
      on line,
      boxsep=0pt,
      left=2.5pt,
      right=2.5pt,
      top=0pt,
      bottom=0pt,
      enlarge top initially by=-14pt,
      enlarge bottom by=-14pt,
      enlarge right by=-4pt,
      enlarge left by=-2pt,
      opacityframe=0,
      colback=githubgrey,
      fontupper={\ttfamily\mystrut},
      fontlower={\ttfamily\mystrut}]
      {\textcolor{black}{\textbf{#2}}}
}
\newcommand{\datakeuze}[2][text]{%
    \tcbox[
      on line,
      boxsep=0pt,
      left=2.5pt,
      right=2.5pt,
      top=0pt,
      bottom=0pt,
      enlarge top initially by=-14pt,
      enlarge bottom by=-14pt,
      enlarge right by=-4pt,
      enlarge left by=-2pt,
      opacityframe=0,
      colback=githubgrey,
      fontupper={\ttfamily\mystrut},
      fontlower={\ttfamily\mystrut}]
      {\textcolor{black}{\textbf{#2}}}
}

\usepackage{MyriadPro}

\begin{document}

%% set default font to myriad pro
%% \renewcommand{\familydefault}{\sfdefault}
\sffamily


\usetikzlibrary{arrows}
\usetikzlibrary{shapes.multipart}

%% TODO
%% - check if different number style in Myriad pro looks nice
%% - try semi boldfont for Attribute names
%% - maybe bullets instead of plusses? Making a proper list also gives more flexibility
%%    - all the more because the plusses serve little purposes

\begin{tikzpicture}[
    %% not sure why the "header" is still centered after align=left, but oh well
    ORI object/.style={rectangle split, rectangle split parts=2,
      draw=black, align=left, rounded corners=5pt, very thick, inner sep=10pt}]
  
  \node (agendapunt) [ORI object]
        {{\large \textbf{Objecttype::Agendapunt}}
          \nodepart[inner sep=-1pt]{two}
          + \; ID: \datatype{string} \\
          + \; Agendapunt omschrijving: \datatype{string} [0..1] \\
          + \; Gepland agendapuntvolgnummer: \datatype{string} [0..1] \\
          + \; Agendapuntvolgnummer: \datatype{string} [0..1] \\
          + \; Geplande starttijd: \datatype{datum + tijd} [0..1] \\
          + \; Geplande eindtijd: \datatype{datum + tijd} [0..1] \\
          + \; Starttijd: \datatype{datum + tijd} [0..1] \\
          + \; Eindtijd: \datatype{datum + tijd} [0..1] \\
          + \; Agendapunt kenmerk: \datatype{string} [0..1] \\
          + \; Agendapunt titel: \datatype{string} [0..1] \\
          + \; Indicatie hamerstuk: \datatype{boolean} [0..1] \\
          + \; Indicator behandeld: \datatype{boolean} [0..1] \\
          + \; Indicator besloten: \datatype{boolean} \\
          + \; Bestuurslaag: \datatype{bestuurslaagGegevens} [0..1] \\
          + \; Stemming type: \datakeuze{Hoofdelijk | Regulier | Schriftelijk} [0..1]
        };
\end{tikzpicture}

\end{document}
