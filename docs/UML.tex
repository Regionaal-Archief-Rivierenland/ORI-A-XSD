\documentclass[margin={-40pt 15pt 15pt 15pt}]{standalone}

\usepackage{tikz}

\usepackage[kerning=true, tracking=true]{microtype}
\usepackage[dutch]{babel}

%% font stuff
\usepackage[lf]{MyriadPro}
%% math font for myriadpro
\usepackage{mdsymbol}
\usepackage[T1]{fontenc}

%% math symols
\usepackage{amsmath}
\usepackage{mathtools}

\usepackage{inconsolata}
%% computer modern monospace?
%% \usepackage[tt={oldstyle=false,variable=false}]{cfr-lm}


%% Color definitions
\usepackage[dvipsnames]{xcolor} \definecolor{githubgrey}{HTML}{EAEEF2}
\definecolor{githubred}{HTML}{F2EAEE}
\definecolor{githubblue}{HTML}{DAE1E8}
\definecolor{darkgreen}{HTML}{3D5E45}
\definecolor{purple}{HTML}{5C5573}
%% rounded boxes
\usepackage[most]{tcolorbox}

%% for templates
\usepackage{xparse}

\begin{document}

%% set default font to myriad pro
%% \renewcommand{\familydefault}{\sfdefault}
\sffamily

\usetikzlibrary{arrows}
\usetikzlibrary{fit}
\usetikzlibrary{calc}
\usetikzlibrary{arrows.meta}
\usetikzlibrary{shapes.multipart}
\usetikzlibrary{positioning}
%% needed for making svgs with white backgrounds
\usetikzlibrary{backgrounds}

%% TODO
%% - be consistent in ur use of baselineskips (and amounts)
%% - : na verplicht attribuut moet eigenlijk ook bold als de attribuutnaam dat is
%% - moeten verplichte relaties ook dik gedrukt/anders getypeset
%% - I think the arrow between vergadering and mediabron should be verplicht at least once
%% - make bestuurslaag a "GegevensGroupType" that recieves multiple arrows
%% - check if different number style in Myriad pro looks nice
%%   - not really
%% - try semi boldfont for Attribute names
%%   - too difficult with pdflatex
%% - maybe bullets instead of plusses? Making a proper list also gives more flexibility
%%    - all the more because the plusses serve little purposes

\newcommand{\naamxshift}{-3cm}
\newcommand\mystrut{\rule[-2.5pt]{0pt}{12pt}}
%% datatype typesetting
\newcommand{\datatype}[2][text]{%
    \tcbox[
      on line,
      boxsep=0pt,
      left=2.5pt,
      right=2.5pt,
      top=0pt,
      bottom=0pt,
      enlarge top initially by=-14pt,
      enlarge bottom by=-14pt,
      enlarge right by=-4pt,
      enlarge left by=-2pt,
      opacityframe=0,
      colback=githubgrey,
      fontupper={\ttfamily\mystrut},
      fontlower={\ttfamily\mystrut}]
      {\textcolor{black}{\textbf{#2}}}
}

\newcommand{\umltype}[1]{
  {\textbf{\textcolor{gray}{\guillemotleft#1\guillemotright}}}
}

%% Header of boxes
\newcommand{\objectheader}[2]{
  \umltype{#1} \\
  {\large \textbf{#2}}
  \nodepart[align=left]{two}
}

%% Attribute typesetting
\newcommand{\attribute}[3]{+ \; #1:\, \datatype{#2} \; {\figureversion{text}\textcolor{gray}{#3}}}

\newcommand{\keuze}[1]{\textcolor{darkgreen}{``#1''}}


\begin{tikzpicture}[
    background rectangle/.style={fill=white}, show background rectangle,
    %% not sure why the "header" is still centered after align=left, but oh well
    %% changerectangle split part fill for colors
    ORI object/.style={rectangle split, rectangle split parts=2, rectangle split part fill={white,white}, draw=black, rounded corners=5pt, very thick, inner sep=10pt},
    ORI relatieklasse/.style={rectangle split, rectangle split parts=2, rectangle split part fill={white,white}, rounded corners=5pt, very thick, inner sep=10pt, draw=gray},
    relatiepijl/.style={-{Straight Barb[length=4mm,width=6mm]}, very thick},
    gegevenspijl/.style={-{Straight Barb[length=4mm,width=6mm]}, very thick, dashed},
    kardinaliteit/.style={text=darkgray}, 
    align=center]
  
  \node (agendapunt) [ORI object, align=center]
        {
          \objectheader{Objecttype}{Agendapunt}

          \attribute{\textbf{ID}}{string}{} \\
          \attribute{Agendapunt omschrijving}{string}{[0..1]} \\
          \attribute{Gepland agendapuntvolgnummer}{string}{[0..1]} \\
          \attribute{Agendapuntvolgnummer}{string}{[0..1]} \\
          \attribute{Geplande starttijd}{datum + tijd}{[0..1]} \\
          \attribute{Geplande eindtijd}{datum + tijd}{[0..1]} \\
          \attribute{Starttijd}{datum + tijd}{[0..1]} \\
          \attribute{Eindtijd}{datum + tijd}{[0..1]} \\
          \attribute{Agendapunt kenmerk}{string}{[0..1]} \\
          \attribute{Agendapunt titel}{string}{[0..1]} \\
          \attribute{Indicatie hamerstuk}{boolean}{[0..1]} \\
          \attribute{Indicator behandeld}{boolean}{[0..1]} \\
          \attribute{\textbf{Indicator besloten}}{boolean}{} \\
          \attribute{Bestuurslaag}{bestuurslaagGegevens}{[0..1]}
        };


   \node (vergadering) [ORI object, align=center, left=14cm of agendapunt]
        {
          \objectheader{Objecttype}{Vergadering}

          \attribute{\textbf{ID}}{string}{} \\
          \attribute{Naam}{string}{[0..1]} \\
          \attribute{Vergadertoelichting}{string}{[0..1]} \\
          \attribute{Georganiseerd door gremium}{gremiumGegevens}{[0..1]} \\
          \attribute{\textbf{Vergaderingstype}}{\keuze{Raadsvergadering} | \keuze{Commissievergadering} | ...}{} \\
          \attribute{Locatie}{string}{[0..1]} \\
          \attribute{Status}{\keuze{Gepland} | \keuze{Gehouden} | \keuze{Geannuleerd}}{[0..1]} \\
          \attribute{Geplande vergaderdatum}{datum}{[0..1]} \\
          \attribute{Vergaderdatum}{datum}{[0..1]} \\
          \attribute{Geplande aanvang vergadering}{datum + tijd}{[0..1]} \\
          \attribute{Gepland einde vergadering}{datum + tijd}{[0..1]} \\
          \attribute{Aanvang vergadering}{datum + tijd}{[0..1]} \\
          \attribute{Einde vergadering}{datum + tijd}{[0..1]} \\
          \attribute{Publicatiedatum}{datum + tijd}{[0..1]} \\
          \attribute{Bestuurslaag}{bestuurslaagGegevens}{[0..1]}
        };

   \node (deelnemer) [ORI object, align=center, below=3.9cm of vergadering]
        {
          \objectheader{Objecttype}{Aanwezige deelnemer}

          \attribute{ID}{string}{[0..1]} \\
          \attribute{\textbf{Rolnaam}}{
            \begin{minipage}[t]{0.55\linewidth}
              \raggedright
              \keuze{Voorzitter} | \keuze{Vice-voorzitter} | \keuze{Raadslid} | \keuze{Statenlid} | \keuze{Dagelijks bestuurslid} | \keuze{Algemeen bestuurslid} | \keuze{Inspreker} | \keuze{Portefeuillehouder} | \keuze{Griffier} | \keuze{Overig} \vspace{0.3em}
            \end{minipage}
          }{} \\[1.2\baselineskip]
          \attribute{Organisatie}{string}{[0..1]} \\
          \attribute{Deelnemerspositie}{string}{[0..1]} \\
          \attribute{Aanvang aanwezigheid}{datum + tijd}{[0..1]} \\
          \attribute{Einde aanwezigheid}{datum + tijd}{[0..1]}
        };

        %% One could argue that this should just be another attribute
        \draw[relatiepijl] ($(vergadering.west)+(0,1.8)$)
        node [kardinaliteit, above, xshift = -0.4cm] {0..1} |-
        ($(vergadering.west)+(-3.0,1.8)$) |- ($(vergadering.west)+(-3.0,-1.8)$)
        node [fill=white, pos=0.25] {heeft als deelvergadering\\\umltype{Relatiesoort}} --
        ($(vergadering.west)+(0,-1.8)$)
        node [kardinaliteit, above, xshift = -0.8cm] {0..*} ;

    \node (stemming) [ORI object, below = 3.9cm of agendapunt, align=center]
        {
          \objectheader{Objecttype}{Stemming}

          \attribute{\textbf{ID}}{string}{} \\
          \attribute{Stemmingstype}{\keuze{Hoofdelijk} | \keuze{Regulier} | \keuze{Schriftelijk}}{[0..1]}\\
          \attribute{Resultaat mondelinge stemming}{\keuze{Voor} | \keuze{Tegen} | \keuze{Gelijk}}{[0..1]} \\
          \attribute{Toezegging}{string}{[0..1]} \\
          \attribute{Resultaat stemming over personen}{string}{[0..1]} \\
          \attribute{Stemming over personen}{stemmingOverPersonenGegevens}{[0..*]}
        };

    \node (fractie) [ORI object, below = 2.9cm of stemming, align=center]
        {
          \objectheader{Objecttype}{Fractie}

          \attribute{ID}{string}{[0..1]} \\
          \attribute{\textbf{Fractienaam}}{string}{}\\
          \attribute{Bestuurslaag}{bestuurslaagGegevens}{[0..1]}
        };
        
   \node (stem) [ORI relatieklasse, align=center, at = ($(deelnemer.east)!0.5!(stemming.west)$), yshift=3.4cm]
        {
          \objectheader{Relatieklasse}{Stem}

          \attribute{ID}{string}{[0..1]} \\
          \attribute{\textbf{Keuze stemming}}{
            \begin{minipage}[t]{0.44\linewidth}
              \raggedright
              \keuze{Voor} | \keuze{Tegen} | \keuze{Afwezig} | \keuze{Onthouden}
              \vspace{0.3em}
            \end{minipage}
          }{}
          \vspace*{0.42cm}
        };
   \node (stemresultaatperfractie) [ORI relatieklasse, align=center, below = 4.2cm of stem]
        {
          \objectheader{Relatieklasse}{Stemresultaat per fractie}

          \attribute{ID}{string}{[0..1]} \\
          \attribute{Fractie stemresultaat}{\keuze{Aangenomen} | \keuze{Verworpen} | \keuze{Verdeeld}}{[0..1]}
        };
   \node (dagelijksbestuur) [ORI object, align=center, below = 7.5cm of fractie]
        {
          \objectheader{Objecttype}{Dagelijks bestuur}

          \attribute{\textbf{ID}}{string}{} \\
          \attribute{\textbf{Dagelijks bestuur naam}}{string}{} \\ 
          \attribute{\textbf{Dagelijks bestuur type}}{\keuze{enumeratie dagelijks bestuur type}}{} \\
          \attribute{\textbf{Bestuurslaag}}{bestuurslaagGegevens}{}
        };
   \node (spreekfragment) [ORI relatieklasse, above = 17.5cm of stem]
        {
          \objectheader{Relatieklasse}{Spreekfragment}

          \attribute{ID}{string}{[0..1]} \\
          \attribute{Aanvang spreekfragment}{datum + tijd}{[0..1]} \\
          \attribute{Einde spreekfragment}{datum + tijd}{[0..1]} \\
          \attribute{Taal}{taal}{[0..1]} \\
          \attribute{Tekst spreekfragment}{string}{[0..1]} \\
          \attribute{Titel spreekfragment}{string}{[0..1]} \\
          \attribute{Positie notulen}{string}{[0..1]} \\
          \attribute{Audio tijdsaanduiding aanvang}{integer}{[0..1]} \\
          \attribute{Audio tijdsaanduiding einde}{integer}{[0..1]} \\
          \attribute{Video tijdsaanduiding aanvang}{integer}{[0..1]} \\
          \attribute{Video tijdsaanduiding einde}{integer}{[0..1]}
        };
        
   \node (natuurlijkpersoon) [ORI object, align=center, left=5.0cm of deelnemer]
        {
          \objectheader{Objecttype}{Natuurlijk persoon}

          \attribute{\textbf{ID}}{string}{} \\
          \attribute{Geslachtsaanduiding}{\keuze{Man} | \keuze{Vrouw} | \keuze{Anders} | \keuze{Onbekend}}{[0..1]} \\[0.4\baselineskip]
          \attribute{Functie}{
            \begin{minipage}[t]{0.72\linewidth}
              \raggedright
              \keuze{Burgemeester} | \keuze{Wethouder} | \keuze{Raadslid} | \keuze{Burgerlid} | \keuze{Griffier} | \keuze{Gemeentesecretaris} | \keuze{Commissaris van de Koning} | \keuze{Gedeputeerde} | \keuze{Statenlid} | \keuze{Provinciesecretaris} | \keuze{Dijkgraaf} | \keuze{Dagelijks bestuurslid} | \keuze{Algemeen bestuurslid} | \keuze{Secretarisdirecteur} | \keuze{Ambtenaar/medewerker} | \keuze{Adviseur of deskundige} | \keuze{Overig} \vspace*{0.3em}
            \end{minipage}
          }{[0..1]} \\[1.2\baselineskip]
          \attribute{\textbf{Naam}}{naamGegevens}{} \\
          \attribute{Nevenfunctie}{nevenfunctieGegevens}{[0..*]}
        };
   \node (naam) [ORI object, below = 2.2cm of natuurlijkpersoon, align=center, draw=gray, xshift=\naamxshift]
        {
          \objectheader{Gegevensgroeptype}{Naam}

          \attribute{\textbf{Achternaam}}{string}{} \\
          \attribute{Tussenvoegsel}{string}{[0..1]} \\
          \attribute{Voorletters}{string}{[0..1]} \\
          \attribute{Voornamen}{string}{[0..1]} \\
          \attribute{Volledige naam}{string}{[0..1]}
        }; 
   \node (nevenfunctie) [ORI object, above left = 2.2cm and -9cm of natuurlijkpersoon, align=center, draw=gray]
        {
          \objectheader{Gegevensgroeptype}{Nevenfunctie}

          \attribute{\textbf{Omschrijving nevenfunctie}}{string}{} \\
          \attribute{Naam organisatie nevenfunctie}{string}{[0..1]} \\
          \attribute{Aantal uren per maand}{integer}{[0..1]} \\
          \attribute{\textbf{Indicator bezoldigd}}{boolean}{} \\
          \attribute{Indicator nevenfunctief vanwege lidmaatschap}{boolean}{[0..1]} \\
          \attribute{Datum melding}{datum}{[0..1]} \\
          \attribute{\textbf{Datum aanvang nevenfunctie}}{datum}{} \\
          \attribute{Datum einde nevenfunctie}{datum}{[0..1]}
        };
   \node (fractielidmaatschap) [ORI relatieklasse, below = 2.2cm of deelnemer, align=center]
        {
          \objectheader{Relatieklasse}{Fractielidmaatschap}

          \attribute{ID}{string}{[0..1]} \\
          \attribute{Datum begin fractielidmaatschap}{datum}{[0..1]} \\
          \attribute{Datum einde fractielidmaatschap}{datum}{[0..1]} \\
          \attribute{\textbf{Indicate voorzitter}}{boolean}{}
        };
   \node (dagelijksbestuurlidmaatschap) [ORI relatieklasse, align=center, below = 3.1cm of fractielidmaatschap]
        {
          \objectheader{Relatieklasse}{Dagelijks bestuur lidmaatschap}

          \attribute{ID}{string}{[0..1]} \\
          \attribute{Datum begin dagelijks bestuur lidmaatschap}{datum}{[0..1]} \\ 
          \attribute{Datum eind dagelijks bestuur lidmaatschap}{datum}{[0..1]}
        };
   \node (mediabron) [ORI object, align=center, above=2.5cm of vergadering]
        {
          \objectheader{Objecttype}{Mediabron}

          \attribute{\textbf{ID}}{string}{} \\
          \attribute{Mimetype}{string}{[0..1]} \\
          \attribute{\textbf{Mediabron type}}{\keuze{Video} | \keuze{Audio} | \keuze{Transcriptie}}{} \\
          \attribute{URL}{URI}{[0..1]}
        };
        

  \node (besluit) [ORI object, above right = 0.5cm and 1.8cm of stemming, align=center]
        {
          \objectheader{Objecttype}{Besluit}

          \attribute{ID}{string}{[0..1]} \\
          \attribute{Besluit toelichting}{string}{[0..1]} \\
          \attribute{Toezegging}{string}{[0..1]} \\
          \attribute{\textbf{Besluit resultaat}}{\keuze{enumeratie Besluit resultaat}}{}
        };

  \node (stemmingOverPersonenGegevens) [ORI object, below right = 5.3cm and -10cm of besluit, align=center, draw=gray]
        {
          \objectheader{Gegevensgroeptype}{Stemming over personen}

          \attribute{\textbf{Naam kandidaat}}{string}{} \\
          \attribute{\textbf{Aantal uitgebrachte stemmen}}{integer}{}
        };

   \node (vergaderstuk) [ORI object, above right = -4.6cm and 5.2cm of agendapunt, align=center]
        {
          \objectheader{Relatieklasse}{Vergaderstuk}

          \attribute{Vergaderstuk type}{\keuze{enumeratie Vergaderstuk type}}{[0..1]}
        };

  %% inner MDTO rectangle
  \node (mdto-inner) [rectangle, draw, ultra thick, right = 3.2cm of vergaderstuk, inner sep=40pt, fill=githubgrey, align=center]
        {
          {\huge MDTO}
        };

  % Outer MDTO rectangle (this is what arrows should connect to)
  \node (mdto) [fit=(mdto-inner), draw, ultra thick, inner sep=1.1cm] {};


        %% arrows
        %% FIXME: cardinalities won't be alligned, because pos= is relative to arrow length
        %% stemming -> agendapunt
        \draw [relatiepijl] (stemming) --
        node [anchor=center, fill=white] {heeft betrekking op\\\umltype{Relatiesoort}}
        node [kardinaliteit, pos=0.08, right] {1..*}
        node [kardinaliteit, pos=0.81, xshift=0.03cm, right] {1} (agendapunt);
        %% stemming -> besluit
        \draw[relatiepijl] ($(stemming.north)+(2.5,0)$) |- (besluit.west)
        node [pos=0.74, above] {leidt tot}
        node [pos=0.74, below] {\umltype{Relatiesoort}}
        node [kardinaliteit, pos=0.065, right] {0..*}
        node [kardinaliteit, above left, pos=0.95, yshift=-0.0cm] {0..1};
        %% stemming -> stemming over personen
        \draw[gegevenspijl] ($(stemming.south) + (2.5, -0.0cm)$) |- (stemmingOverPersonenGegevens.west);
        %% stemming -> MDTO
        \draw[relatiepijl] ($(stemming.east) + (0, 0.0cm)$) -|
        node [pos=0.330, above] {heeft betrekking op}
        node [pos=0.330, below] {\umltype{Relatiesoort}}
        coordinate[pos=0.330] (tempcoordinate) % coordinate along the line
        (mdto.south);
        %% draw dashed line to vergaderstuk
        \coordinate (lineendnorth) at (tempcoordinate|-vergaderstuk.south);
        \draw[dashed, very thick] ($(tempcoordinate) + (0, 0.5cm)$) -- (lineendnorth);
        
        %% agendapunt -> vergadering
        \draw[relatiepijl] (agendapunt) -- (vergadering)
        node [pos=0.5, above] {wordt behandeld tijdens}
        node [pos=0.5, below] {\umltype{Relatiesoort}}
        node [kardinaliteit, pos=0.05, above] {0..*}
        node [kardinaliteit, pos=0.9, above] {1};

        %% agendapunt -> mdto
        \draw[relatiepijl] 
        ($(agendapunt.east) - (0, 2.5cm)$) -- ++(2cm,0) -| ($(mdto.south) - (2cm, 0cm)$)
        coordinate[pos=0.24] (tempcoordinate)
        node [pos=0.24, above] {heeft als bijlage}
        node [pos=0.24, below] {\umltype{Relatiesoort}};
        % Draw the dashed line
        \coordinate (lineendnorth) at (tempcoordinate|-vergaderstuk.south);
        \draw[dashed, very thick] ($(tempcoordinate) + (0, 0.52cm)$) -- (lineendnorth);

        %% mediabron -> mdto (is vastgelegd in)
        \draw[relatiepijl] (mediabron) ($(mediabron.east)+(0,0.3)$) -|
        node [pos=0.18, above] {is vastgelegd in}
        node [pos=0.18, below] {\umltype{Relatiesoort}}
        coordinate[pos=0.22] (linestart) % coordinate along the line
        ($(mdto.north)+(2.8cm,0)$);

        %% mediabron -> mdto (ondertitel bestand)
        \draw[relatiepijl] (mediabron) ($(mediabron.east)-(0,1.3)$) -|
        node [pos=0.23, above] {heeft ondertitelbestand}
        node [pos=0.23, below] {\umltype{Relatiesoort}}
        coordinate[pos=0.26] (linestart4) % coordinate along the line
        ($(mdto.north)+(1cm, 0)$);

        %% fractie -> stemming
        \draw[relatiepijl] (fractie) --
        coordinate[pos=0.5] (tempcoordinate) %% define tmp halfway coord (for allignment reasons)
        (stemming);
        %% needed for allignment reasons
        \coordinate (linestartfractiestemresultaat) at (tempcoordinate |- stemresultaatperfractie.east);
        %% dashed line to stemresultaat per fractie
        \draw[dashed, very thick] (linestartfractiestemresultaat) --
        node [pos=0, anchor=center, fill=white] {neemt deel aan\\\umltype{Relatiesoort}}
        (stemresultaatperfractie.east);
        
        
        %% vergadering -> mediabron
        \draw[relatiepijl] (vergadering)  -- 
        node [anchor=center, fill=white] {is vastgelegd middels\\\umltype{Relatiesoort}} (mediabron);

        %% vergadering -> mdto (bijlage)
        \draw[relatiepijl] ($(vergadering.east)+(0, 3)$) -|
        ($(vergadering.east)+(11, 5.5)$) -| 
        node [anchor=center, above, pos=0.3] {heeft als bijlage}
        node [anchor=center, below, pos=0.3] {\umltype{Relatiesoort}}
        coordinate[pos=0.3] (tempcoordinate) %% define tmp halfway coord
        ($(mdto.north)+(-2.8cm,0)$);

        % Draw the dashed line
        \coordinate (lineendsouth) at (tempcoordinate|-vergaderstuk.north);
        \draw[dashed, very thick] ($(tempcoordinate) + (0, -0.5cm)$) -- (lineendsouth);
        
        %% vergadering -> mdto (notulen)
        \draw[relatiepijl] ($(vergadering.north)+(4, 0)$) --
        ($(vergadering.north)+(4, 2)$) -| 
        node [anchor=center, above, pos=0.3] {is genotuleerd in}
        node [anchor=center, below, pos=0.3] {\umltype{Relatiesoort}}
        ($(mdto.north)+(-1cm,0)$);
        
       %% deelnemer -> vergadering
        \draw[relatiepijl] (deelnemer)  -- 
        node [anchor=center, fill=white] {neemt deel aan\\\umltype{Relatiesoort}} (vergadering);
        %% deelnemer -> persoon
        \draw[relatiepijl] (deelnemer) --
        node [above] {is natuurlijk persoon}
        node [below] {\umltype{Relatiesoort}}
        (natuurlijkpersoon);
        %% deelnemer -> stemming
        \draw[dashed, very thick] ($(deelnemer.east)!0.5!(stemming.west) + (0, .54)$) -- (stem);
        \draw[relatiepijl] ($(deelnemer.east |- stemming.west)$) --
        node [above, fill=white] {neemt deel aan} %% no coordinates set (problematic!)
        node [below] {\umltype{Relatiesoort}} %% no coordinates set (problematic!)
        (stemming.west);
        %% deelnemer -> agendapunt (by way of spreekfragment)
        \draw[relatiepijl] ($(deelnemer.north) + (3cm, 0)$) -- ($(deelnemer.north) + (3cm, 3.0cm)$) --
        coordinate[pos=0.9] (tempcoordinate) %% define tmp halfway coord (for allignment reasons)
        ($(deelnemer.north) + (12cm, 3.0cm)$) |- ($(agendapunt.west) -  (0, 3.0cm)$);

        
        %% fixme: this should be perpandiculair
        % Define linestartspreekfragment, horizontally shifted by 1cm from spreekfragment.south 
        % but vertically aligned with tempcoordinate (line X)
        \coordinate (linestartspreekfragment) at ($(tempcoordinate) + (-1.5cm, 0)$);

        % Define endpoint of the dashed line, horizontally shifted by 1cm from spreekfragment.south
        \coordinate (lineendsouth) at ($(spreekfragment.south) + (-1.5cm, 0)$);

        
        % Draw the dashed line
        \draw[dashed, very thick] 
        (linestartspreekfragment) -- 
        node [pos=0.0, anchor=center, fill=white] {spreekt tijdens\\\umltype{Relatiesoort}}
        (lineendsouth);


        %% persoon -> naam
        \draw[gegevenspijl] ([xshift=-3cm] natuurlijkpersoon.south) -- ([xshift=-3cm] naam);
        %% persoon -> nevenfunctie
        \draw[gegevenspijl] ([xshift=-3cm] natuurlijkpersoon.north) -- ([xshift=-3cm] natuurlijkpersoon.north |- nevenfunctie.south);
        %% persoon -> fractie (by way of fractielidmaatschap)
        \draw[relatiepijl] (natuurlijkpersoon) ($(natuurlijkpersoon.south)+(4.9,0)$)  |-
        coordinate[pos=1.0] (linestartfractielidmaatschap) % coordinate along the line
        ($(fractielidmaatschap.south)-(0.0cm,1.85cm)$) -| (fractie.south);
        %% dashed line
        \draw[dashed, very thick] (linestartfractielidmaatschap) --
        node [above, fill=white, pos=0.0] {is lid van}
        node [below, pos=0.0] {\umltype{Relatiesoort}}
        (fractielidmaatschap.south);
        %% persoon -> dagelijks bestuur (by way of dagelijksbestuurlidmaatschap)
        \draw[relatiepijl] (natuurlijkpersoon) ($(natuurlijkpersoon.south)+(2.45,0)$)  |-
        ($(dagelijksbestuur.west) - (0, 1.0cm)$);
        %% dashed line
        \draw[dashed, very thick] (dagelijksbestuurlidmaatschap.south) --
        node [above, fill=white, pos=1.05] {is lid van}
        node [below, pos=1.05] {\umltype{Relatiesoort}}
        ($(dagelijksbestuurlidmaatschap.south) - (0, 1.8cm)$);

        %% spreekfragment -> mediabron
        \draw[relatiepijl] (spreekfragment.west) -|
        node [above, fill=white, pos=0.27] {is vastgelegd in}
        node [below, pos=0.27] {\umltype{Relatiesoort}}
        (mediabron.north);
        
\end{tikzpicture}

\end{document}
