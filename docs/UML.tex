\documentclass[margin={-40pt 15pt 15pt 15pt}]{standalone}

\usepackage{tikz}

\usepackage[kerning=true, tracking=true]{microtype}
\usepackage[dutch]{babel}

%% font stuff
\usepackage[lf]{MyriadPro}
%% math font for myriadpro
\usepackage{mdsymbol}
\usepackage[T1]{fontenc}
\usepackage{varwidth}

%% math symols
\usepackage{amsmath}
\usepackage{mathtools}

%% Tables
\usepackage{makecell}
\usepackage{colortbl} % To enable \arrayrulecolor

\usepackage{inconsolata}
%% computer modern monospace?
%% \usepackage[tt={oldstyle=false,variable=false}]{cfr-lm}

%% linestretching
\usepackage{setspace}

%% rounded boxes
\usepackage[most]{tcolorbox}

%% for templates
\usepackage{xparse}

%% Color definitions
\usepackage[dvipsnames, table]{xcolor}
\definecolor{githubgrey}{HTML}{EAEEF2}
\definecolor{githubred}{HTML}{F2EAEE}
\definecolor{mainblack}{HTML}{333333}
\definecolor{blueishblack}{HTML}{3A3A3A}
\definecolor{lightgray}{HTML}{D9DDE1}
\definecolor{githubblue}{HTML}{DAE1E8}
\definecolor{darkgreen}{HTML}{3D5E45}
\definecolor{purple}{HTML}{5C5573}
\definecolor{enumeratiecolor}{HTML}{3C5770}

\begin{document}

%% set default font to myriad pro
%% \renewcommand{\familydefault}{\sfdefault}
\sffamily

\usetikzlibrary{arrows}
\usetikzlibrary{fit}
\usetikzlibrary{calc}
\usetikzlibrary{arrows.meta}
\usetikzlibrary{shapes.multipart}
\usetikzlibrary{shapes.misc}
\usetikzlibrary{positioning}
%% needed for making svgs with white backgrounds
\usetikzlibrary{backgrounds}

%% TODO
%% - be consistent in ur use of baselineskips (and amounts)
%% - : na verplicht attribuut moet eigenlijk ook bold als de attribuutnaam dat is
%% - moeten verplichte relaties ook dik gedrukt/anders getypeset
%% - I think the arrow between vergadering and mediabron should be verplicht at least once
%% - make bestuurslaag a "GegevensGroupType" that recieves multiple arrows
%% - check if different number style in Myriad pro looks nice
%%   - not really
%% - try semi boldfont for Attribute names
%%   - too difficult with pdflatex
%% - maybe bullets instead of plusses? Making a proper list also gives more flexibility
%%    - all the more because the plusses serve little purposes

\newcommand{\naamxshift}{-3cm}
\newcommand\mystrut{\rule[-2.5pt]{0pt}{12pt}}
%% datatype typesetting
\newcommand{\datatype}[2][text]{%
    \tcbox[
      on line,
      boxsep=0pt,
      left=2.5pt,
      right=2.5pt,
      top=0pt,
      bottom=0pt,
      enlarge top initially by=-14pt,
      enlarge bottom by=-14pt,
      enlarge right by=-4pt,
      enlarge left by=-2pt,
      opacityframe=0,
      colback=githubgrey,
      fontupper={\ttfamily\mystrut},
      fontlower={\ttfamily\mystrut}]
      {\textcolor{blueishblack}{\textbf{#2}}}
}

\newcommand{\umltype}[1]{
  {\textbf{\textcolor{gray}{\guillemotleft#1\guillemotright}}}
}

\newcommand{\enumeratietype}[1]{enumeratie \textcolor{darkgreen}{#1}}

%% Header of boxes
\newcommand{\objectheader}[2]{
  \umltype{#1} \\
  {\large \textcolor{mainblack}{\textbf{#2}}}
  \nodepart[align=left]{two}
}

%% TODO: try if a regular box looks better (maybe make the whole box a certain color? or just the bottom half)
%% Attribute typesetting
\newcommand{\attribute}[3]{\textcolor{mainblack}{+ \; #1:}\, \datatype{#2} \; {\figureversion{text}\textcolor{gray}{#3}}}


\begin{tikzpicture}[
    %% needed to fix svg
    background rectangle/.style={fill=white}, show background rectangle,
    %% not sure why the "header" is still centered after align=left, but oh well
    %% changerectangle split part fill for colors
    ORI object/.style={rectangle split, rectangle split parts=2, rectangle split part fill={white,white}, draw=mainblack, rounded corners=5pt, very thick, inner sep=10pt},
    ORI relatieklasse/.style={rectangle split, rectangle split parts=2, rectangle split part fill={white,white}, rounded corners=5pt, very thick, inner sep=10pt, draw=gray!90},
    relatiepijl/.style={-{Straight Barb[length=4mm,width=6mm]}, very thick, draw=mainblack},
    gegevenspijl/.style={-{Straight Barb[length=4mm,width=6mm]}, very thick, dashed, draw=mainblack},
    kardinaliteit/.style={text=darkgray}, 
    align=center]
  
  \node (agendapunt) [ORI object, align=center]
        {
          \objectheader{Objecttype}{Agendapunt}

          \attribute{\textbf{ID}}{string}{} \\
          \attribute{Agendapunt omschrijving}{string}{[0..1]} \\
          \attribute{Gepland agendapuntvolgnummer}{string}{[0..1]} \\
          \attribute{Agendapuntvolgnummer}{string}{[0..1]} \\
          \attribute{Geplande starttijd}{datum + tijd}{[0..1]} \\
          \attribute{Geplande eindtijd}{datum + tijd}{[0..1]} \\
          \attribute{Starttijd}{datum + tijd}{[0..1]} \\
          \attribute{Eindtijd}{datum + tijd}{[0..1]} \\
          \attribute{Agendapunt kenmerk}{string}{[0..1]} \\
          \attribute{Agendapunt titel}{string}{[0..1]} \\
          \attribute{Indicatie hamerstuk}{boolean}{[0..1]} \\
          \attribute{Indicator behandeld}{boolean}{[0..1]} \\
          \attribute{\textbf{Indicator besloten}}{boolean}{} \\
          \attribute{Bestuurslaag}{bestuurslaagGegevens}{[0..1]}
        };


   \node (vergadering) [ORI object, align=center, left=17cm of agendapunt]
        {
          \objectheader{Objecttype}{Vergadering}

          \attribute{\textbf{ID}}{string}{} \\
          \attribute{Naam}{string}{[0..1]} \\
          \attribute{Vergadertoelichting}{string}{[0..1]} \\
          \attribute{Georganiseerd door gremium}{gremiumGegevens}{[0..1]} \\
          \attribute{\textbf{Vergaderingstype}}{\enumeratietype{vergaderingstype}}{} \\
          \attribute{Locatie}{string}{[0..1]} \\
          \attribute{Status}{\enumeratietype{status}}{[0..1]} \\
          \attribute{Geplande vergaderdatum}{datum}{[0..1]} \\
          \attribute{Vergaderdatum}{datum}{[0..1]} \\
          \attribute{Geplande aanvang vergadering}{datum + tijd}{[0..1]} \\
          \attribute{Gepland einde vergadering}{datum + tijd}{[0..1]} \\
          \attribute{Aanvang vergadering}{datum + tijd}{[0..1]} \\
          \attribute{Einde vergadering}{datum + tijd}{[0..1]} \\
          \attribute{Publicatiedatum}{datum + tijd}{[0..1]} \\
          \attribute{Bestuurslaag}{bestuurslaagGegevens}{[0..1]}
        };

   \node (deelnemer) [ORI object, align=center, below=3.9cm of vergadering]
        {
          \objectheader{Objecttype}{Aanwezige deelnemer}

          \attribute{ID}{string}{[0..1]} \\
          \attribute{\textbf{Rolnaam}}{\enumeratietype{rolnaam}}{} \\
          \attribute{Organisatie}{string}{[0..1]} \\
          \attribute{Deelnemerspositie}{string}{[0..1]} \\
          \attribute{Aanvang aanwezigheid}{datum + tijd}{[0..1]} \\
          \attribute{Einde aanwezigheid}{datum + tijd}{[0..1]}
        };

    \node (stemming) [ORI object, below = 3.9cm of agendapunt, align=center]
        {
          \objectheader{Objecttype}{Stemming}

          \attribute{\textbf{ID}}{string}{} \\
          \attribute{Stemmingtype}{\enumeratietype{stemmingtype}}{[0..1]}\\
          \attribute{Resultaat mondelinge stemming}{\enumeratietype{resultaatMondelingeStemming}}{[0..1]} \\
          \attribute{Toezegging}{string}{[0..1]} \\
          \attribute{Resultaat stemming over personen}{string}{[0..1]} \\
          \attribute{Stemming over personen}{stemmingOverPersonenGegevens}{[0..*]}
        };

    \node (fractie) [ORI object, below = 8.5cm of stemming, align=center]
        {
          \objectheader{Objecttype}{Fractie}

          \attribute{ID}{string}{[0..1]} \\
          \attribute{\textbf{Fractienaam}}{string}{}\\
          \attribute{Bestuurslaag}{bestuurslaagGegevens}{[0..1]}
        };
        
   \node (stem) [ORI relatieklasse, align=center, at = ($(deelnemer.east)!0.5!(stemming.west)$), yshift=2.8cm]
        {
          \objectheader{Relatieklasse}{Stem}

          \attribute{ID}{string}{[0..1]} \\
          \attribute{\textbf{Keuze stemming}}{\enumeratietype{keuzeStemming}}{}
        };
   \node (stemresultaatperfractie) [ORI relatieklasse, align=center, below right = 6.5cm and -4.5cm of stem]
        {
          \objectheader{Relatieklasse}{Stemresultaat per fractie}

          \attribute{ID}{string}{[0..1]} \\
          \attribute{Fractie stemresultaat}{\enumeratietype{fractieStemresultaat}}{[0..1]}
        };
   \node (dagelijksbestuur) [ORI object, align=center, below = 5.2cm of fractie]
        {
          \objectheader{Objecttype}{Dagelijks bestuur}

          \attribute{\textbf{ID}}{string}{} \\
          \attribute{\textbf{Dagelijks bestuur naam}}{string}{} \\ 
          \attribute{\textbf{Dagelijks bestuur type}}{\enumeratietype{dagelijksBestuurType}}{} \\
          \attribute{\textbf{Bestuurslaag}}{bestuurslaagGegevens}{}
        };
   \node (spreekfragment) [ORI relatieklasse, above = 17.5cm of stem]
        {
          \objectheader{Relatieklasse}{Spreekfragment}

          \attribute{ID}{string}{[0..1]} \\
          \attribute{Aanvang spreekfragment}{datum + tijd}{[0..1]} \\
          \attribute{Einde spreekfragment}{datum + tijd}{[0..1]} \\
          \attribute{Taal}{taal}{[0..1]} \\
          \attribute{Tekst spreekfragment}{string}{[0..1]} \\
          \attribute{Titel spreekfragment}{string}{[0..1]} \\
          \attribute{Positie notulen}{string}{[0..1]} \\
          \attribute{Audio tijdsaanduiding aanvang}{integer}{[0..1]} \\
          \attribute{Audio tijdsaanduiding einde}{integer}{[0..1]} \\
          \attribute{Video tijdsaanduiding aanvang}{integer}{[0..1]} \\
          \attribute{Video tijdsaanduiding einde}{integer}{[0..1]}
        };
        
   \node (natuurlijkpersoon) [ORI object, align=center, left=5.0cm of deelnemer]
        {
          \objectheader{Objecttype}{Natuurlijk persoon}

          \attribute{\textbf{ID}}{string}{} \\
          \attribute{Geslachtsaanduiding}{\enumeratietype{geslachtsaanduiding}}{[0..1]} \\
          \attribute{Functie}{\enumeratietype{functie}}{[0..1]} \\
          \attribute{\textbf{Naam}}{naamGegevens}{} \\
          \attribute{Nevenfunctie}{nevenfunctieGegevens}{[0..*]}
        };
   \node (naam) [ORI object, below = 2.2cm of natuurlijkpersoon, align=center, draw=gray, xshift=\naamxshift]
        {
          \objectheader{Gegevensgroeptype}{Naam}

          \attribute{\textbf{Achternaam}}{string}{} \\
          \attribute{Tussenvoegsel}{string}{[0..1]} \\
          \attribute{Voorletters}{string}{[0..1]} \\
          \attribute{Voornamen}{string}{[0..1]} \\
          \attribute{Volledige naam}{string}{[0..1]}
        }; 
   \node (nevenfunctie) [ORI object, above left = 2.2cm and -9cm of natuurlijkpersoon, align=center, draw=gray]
        {
          \objectheader{Gegevensgroeptype}{Nevenfunctie}

          \attribute{\textbf{Omschrijving nevenfunctie}}{string}{} \\
          \attribute{Naam organisatie nevenfunctie}{string}{[0..1]} \\
          \attribute{Aantal uren per maand}{integer}{[0..1]} \\
          \attribute{\textbf{Indicator bezoldigd}}{boolean}{} \\
          \attribute{Indicator nevenfunctief vanwege lidmaatschap}{boolean}{[0..1]} \\
          \attribute{Datum melding}{datum}{[0..1]} \\
          \attribute{\textbf{Datum aanvang nevenfunctie}}{datum}{} \\
          \attribute{Datum einde nevenfunctie}{datum}{[0..1]}
        };
   \node (fractielidmaatschap) [ORI relatieklasse, below = 4.0cm of deelnemer, align=center]
        {
          \objectheader{Relatieklasse}{Fractielidmaatschap}

          \attribute{ID}{string}{[0..1]} \\
          \attribute{Datum begin fractielidmaatschap}{datum}{[0..1]} \\
          \attribute{Datum einde fractielidmaatschap}{datum}{[0..1]} \\
          \attribute{\textbf{Indicate voorzitter}}{boolean}{}
        };
   \node (dagelijksbestuurlidmaatschap) [ORI relatieklasse, align=center, below = 5.85cm of fractielidmaatschap]
        {
          \objectheader{Relatieklasse}{Dagelijks bestuur lidmaatschap}

          \attribute{ID}{string}{[0..1]} \\
          \attribute{Datum begin dagelijks bestuur lidmaatschap}{datum}{[0..1]} \\
          \attribute{Datum eind dagelijks bestuur lidmaatschap}{datum}{[0..1]}
        };
   \node (mediabron) [ORI object, align=center, above=2.5cm of vergadering]
        {
          \objectheader{Objecttype}{Mediabron}

          \attribute{\textbf{ID}}{string}{} \\
          \attribute{Mimetype}{string}{[0..1]} \\
          \attribute{\textbf{Mediabron type}}{\enumeratietype{mediabronType}}{} \\
          \attribute{URL}{URI}{[0..1]}
        };
        

  \node (besluit) [ORI object, above right = 0.5cm and 1.8cm of stemming, align=center]
        {
          \objectheader{Objecttype}{Besluit}

          \attribute{ID}{string}{[0..1]} \\
          \attribute{Besluit toelichting}{string}{[0..1]} \\
          \attribute{Toezegging}{string}{[0..1]} \\
          \attribute{\textbf{Besluit resultaat}}{\enumeratietype{besluitResultaat}}{}
        };

  \node (stemmingOverPersonenGegevens) [ORI object, below right = 5.3cm and -10cm of besluit, align=center, draw=gray]
        {
          \objectheader{Gegevensgroeptype}{Stemming over personen}

          \attribute{\textbf{Naam kandidaat}}{string}{} \\
          \attribute{\textbf{Aantal uitgebrachte stemmen}}{integer}{}
        };

   \node (vergaderstuk) [ORI relatieklasse, above right = -4.6cm and 7.2cm of agendapunt, align=center]
        {
          \objectheader{Relatieklasse}{Vergaderstuk}

          \attribute{Vergaderstuk type}{\enumeratietype{vergaderstukType}}{[0..1]}
        };

  %% inner MDTO rectangle
  \node (mdto-inner) [rectangle, draw, ultra thick, right = 3.2cm of vergaderstuk, inner sep=40pt, fill=githubgrey, align=center]
        {
          {\huge MDTO}
        };

  % Outer MDTO rectangle (this is what arrows should connect to)
  \node (mdto) [fit=(mdto-inner), draw, ultra thick, inner sep=1.1cm] {};


  %% arrows
  
        %% FIXME: cardinalities won't be alligned, because pos= is relative to arrow length
        %% stemming -> agendapunt
        \draw [relatiepijl] (stemming) --
        node [anchor=center, fill=white] {heeft betrekking op\\\umltype{Relatiesoort}}
        node [kardinaliteit, pos=0.08, right] {1..*}
        node [kardinaliteit, pos=0.81, xshift=0.03cm, right] {1} (agendapunt);
        %% stemming -> besluit
        \draw[relatiepijl] ($(stemming.north)+(2.5,0)$) |- (besluit.west)
        node [pos=0.74, above] {leidt tot}
        node [pos=0.74, below] {\umltype{Relatiesoort}}
        node [kardinaliteit, pos=0.065, right] {0..*}
        node [kardinaliteit, above left, pos=0.95, yshift=-0.0cm] {0..1};
        %% stemming -> stemming over personen
        \draw[gegevenspijl] ($(stemming.south) + (2.5, -0.0cm)$) |- (stemmingOverPersonenGegevens.west);
        %% stemming -> MDTO
        \draw[relatiepijl] ($(stemming.east) + (0, 0.0cm)$) -|
        node [pos=0.330, above] {heeft betrekking op}
        node [pos=0.330, below] {\umltype{Relatiesoort}}
        coordinate[pos=0.330] (tempcoordinate) % coordinate along the line
        (mdto.south);
        %% draw dashed line to vergaderstuk
        \coordinate (lineendnorth) at (tempcoordinate|-vergaderstuk.south);
        \draw[dashed, very thick] ($(tempcoordinate) + (0, 0.5cm)$) -- (lineendnorth);

        %% deelnemer -> vergadering
        \draw[relatiepijl] (deelnemer)  -- 
        node [anchor=center, fill=white] {neemt deel aan\\\umltype{Relatiesoort}} (vergadering);
        %% deelnemer -> persoon
        \draw[relatiepijl] (deelnemer) --
        node [above] {is natuurlijk persoon}
        node [below] {\umltype{Relatiesoort}}
        (natuurlijkpersoon);
        %% deelnemer -> stemming
        \draw[dashed, very thick] ($(deelnemer.east)!0.5!(stemming.west) + (0, -.19)$) -- (stem);
        \coordinate (deelnemershifted) at ($(deelnemer.east) - (0,0.5cm)$);
        \coordinate (stemmingshifted) at ($(stemming.west) - (0,0.5cm)$);
        % Use the `|-` operator to align vertically while shifting horizontally
        \draw[relatiepijl] 
        ($(deelnemershifted |- stemmingshifted)$) -- 
        node [above, fill=white] {neemt deel aan} 
        node [below] {\umltype{Relatiesoort}} 
        (stemmingshifted.west);
        %% deelnemer -> agendapunt (by way of spreekfragment)
        \draw[relatiepijl] ($(deelnemer.north) + (3cm, 0)$) -- ($(deelnemer.north) + (3cm, 3.0cm)$) --
        coordinate[pos=0.75] (tempcoordinate) %% define relatie line starting position
        ($(deelnemer.north) + (12cm, 3.0cm)$) |- ($(agendapunt.west) -  (0, 3.0cm)$);        
        % Draw the dashed line
        \draw[dashed, very thick] (tempcoordinate) -- 
        node [pos=0.0, anchor=center, fill=white] {spreekt tijdens\\\umltype{Relatiesoort}}
        (tempcoordinate |- spreekfragment.south);

        
        %% agendapunt -> vergadering
        \draw[relatiepijl] (agendapunt) -- (vergadering)
        node [pos=0.5, above] {wordt behandeld tijdens}
        node [pos=0.5, below] {\umltype{Relatiesoort}}
        node [kardinaliteit, pos=0.05, above] {0..*}
        node [kardinaliteit, pos=0.9, above] {1};
        %% agendapunt -> agendapunt
        \draw[relatiepijl] ($(agendapunt.east)+(0, 2.8)$)
        node [kardinaliteit, above, xshift = 0.4cm] {0..1} |-
        ($(agendapunt.east)+(3.0, 2.8)$) |- ($(agendapunt.east)+(3.0, -0.8)$)
        node [fill=white, pos=0.25] {heeft als subagendapunt\\\umltype{Relatiesoort}} --
        ($(agendapunt.east)+(0, -0.8)$)
        node [kardinaliteit, above, xshift = 0.8cm] {0..*};
        %% agendapunt -> mdto
        \draw[relatiepijl] 
        ($(agendapunt.east) - (0, 2.5cm)$) -- ++(2cm,0) -| ($(mdto.south) - (2cm, 0cm)$)
        coordinate[pos=0.27] (tempcoordinate)
        node [pos=0.27, above] {heeft als bijlage}
        node [pos=0.27, below] {\umltype{Relatiesoort}};
        % Draw the dashed line
        \coordinate (lineendnorth) at (tempcoordinate|-vergaderstuk.south);
        \draw[dashed, very thick] ($(tempcoordinate) + (0, 0.52cm)$) -- (lineendnorth);
        %% agendapunt -> natuurlijkpersoon
        %% halfwaypoint stem and stemming
        \coordinate (midpoint) at ($(stem.east)!.5!(stemming.west) + (-0.5cm,0)$);
        % Project the midpoint to the Y-coordinate of stem.north
        \coordinate (midpointadjusted) at (midpoint|-stem.north);
        \draw[very thick, draw=mainblack] ($(agendapunt.south) - (3.0cm,0)$) |-
        (midpointadjusted) --
        ($(midpointadjusted) - (0cm, 4.5cm)$)
        % save bump location
        coordinate[pos=1.13] (bumpstart); % Save the starting point for the bump
         % Create the bump
        \draw[very thick, draw=mainblack] (bumpstart)
        arc[start angle=270, end angle=90, radius=0.3cm];
        \draw[relatiepijl, yshift=-1.2cm] ($(bumpstart) + (0,0.02cm)$) -- ($(bumpstart) - (0,3.3cm)$) -|
        node [pos=0.17, above] {heeft behandelend ambtenaar}
        node [pos=0.17, below] {\umltype{Relatiesoort}}
        ($(natuurlijkpersoon.south) + (3.9cm, 0)$);
        
        %% mediabron -> mdto (is vastgelegd in)
        \draw[relatiepijl] (mediabron) ($(mediabron.east)+(-0.0,0.3)$) -|
        node [pos=0.21, above] {is vastgelegd in}
        node [pos=0.21, below] {\umltype{Relatiesoort}}
        ($(mdto.north)+(2.8cm,0)$);

        %% mediabron -> mdto (ondertitel bestand)
        \draw[relatiepijl] (mediabron) ($(mediabron.east)-(0.0,1.3)$) -|
        node [pos=0.26, above] {heeft ondertitelbestand}
        node [pos=0.26, below] {\umltype{Relatiesoort}}
        ($(mdto.north)+(1cm, 0)$);

        %% fractie -> stemming
        \draw[relatiepijl] (fractie) --
        coordinate[pos=0.5] (tempcoordinate) %% define tmp halfway coord (for allignment reasons)
        (stemming);
        \coordinate (linestartfractiestemresultaat) at (tempcoordinate |- stemresultaatperfractie.east);
        %% dashed line to stemresultaat per fractie
        \draw[dashed, very thick] ($(linestartfractiestemresultaat)-(0,0.3cm)$) --
        node [pos=0, anchor=center, fill=white] {neemt deel aan\\\umltype{Relatiesoort}}
        ($(stemresultaatperfractie.east)-(0,0.3cm)$);
        
        
        %% vergadering -> mediabron
        \draw[relatiepijl] (vergadering)  -- 
        node [anchor=center, fill=white] {is vastgelegd middels\\\umltype{Relatiesoort}} (mediabron);

        %% One could argue that this should just be another attribute
        %% vergadering -> vergadering
        \draw[relatiepijl] ($(vergadering.west)+(0,2.8)$)
        node [kardinaliteit, above, xshift = -0.4cm] {0..1} |-
        ($(vergadering.west)+(-3.0,2.8)$) |- ($(vergadering.west)+(-3.0,-0.8)$)
        node [fill=white, pos=0.25] {heeft als deelvergadering\\\umltype{Relatiesoort}} --
        ($(vergadering.west)+(0,-0.8)$)
        node [kardinaliteit, above, xshift = -0.8cm] {0..*} ;

        %% vergadering -> mdto (bijlage)
        \draw[relatiepijl] ($(vergadering.east)+(0, 3)$) -|
        ($(vergadering.east)+(11, 5.5)$) -| 
        node [anchor=center, above, pos=0.33] {heeft als bijlage}
        node [anchor=center, below, pos=0.33] {\umltype{Relatiesoort}}
        coordinate[pos=0.33] (tempcoordinate) %% define tmp halfway coord
        ($(mdto.north)+(-2.8cm,0)$);

        % Draw the dashed line
        \coordinate (lineendsouth) at (tempcoordinate|-vergaderstuk.north);
        \draw[dashed, very thick] ($(tempcoordinate) + (0, -0.5cm)$) -- (lineendsouth);
        
        %% vergadering -> mdto (notulen)
        \draw[relatiepijl] ($(vergadering.north)+(4, 0)$) --
       ($(vergadering.north)+(4, 2)$) -| 
        node [anchor=center, above, pos=0.32] {is genotuleerd in}
        node [anchor=center, below, pos=0.32] {\umltype{Relatiesoort}}
        ($(mdto.north)+(-1cm,0)$);
        

        %% persoon -> naam
        \draw[gegevenspijl] ([xshift=-3cm] natuurlijkpersoon.south) -- ([xshift=-3cm] naam);
        %% persoon -> nevenfunctie
        \draw[gegevenspijl] ([xshift=-3cm] natuurlijkpersoon.north) -- ([xshift=-3cm] natuurlijkpersoon.north |- nevenfunctie.south);
        %% persoon -> fractie (by way of fractielidmaatschap)
        % calculate endpoint
        \coordinate (Y) at ($(fractielidmaatschap.south)-(0.0cm,2.5cm)$);
        \coordinate (endpoint) at (fractie.west|-Y);
        \draw[relatiepijl] (natuurlijkpersoon) ($(natuurlijkpersoon.south)+(2.6,0)$)  |-
        coordinate[pos=1.0] (linestartfractielidmaatschap) % coordinate along the line
        ($(fractielidmaatschap.south)-(0.0cm,2.5cm)$) -- (endpoint.west);
        %% dashed line
        \draw[dashed, very thick] (linestartfractielidmaatschap) --
        node [above, fill=white, pos=0.0] {is lid van}
        node [below, pos=0.0] {\umltype{Relatiesoort}}
        (fractielidmaatschap.south);
        %% persoon -> dagelijks bestuur (by way of dagelijksbestuurlidmaatschap)
        \draw[relatiepijl] (natuurlijkpersoon) ($(natuurlijkpersoon.south)+(1.2,0)$) |-
        ($(dagelijksbestuur.west) - (0, 0.5cm)$);
        %% dashed line
        \draw[dashed, very thick] (dagelijksbestuurlidmaatschap.south) --
        node [above, fill=white, pos=1.05] {is lid van}
        node [below, pos=1.05] {\umltype{Relatiesoort}}
        ($(dagelijksbestuurlidmaatschap.south) - (0, 1.8cm)$);

        %% spreekfragment -> mediabron
        \draw[relatiepijl] (spreekfragment.west) -|
        node [above, fill=white, pos=0.27] {is vastgelegd in}
        node [below, pos=0.27] {\umltype{Relatiesoort}}
        (mediabron.north);

        %% enumeraties

        \newcommand{\enumeration}[2]{
          \umltype{Enumeratie} \; \textcolor{mainblack}{\textbf{#1}}\\[0.9ex]
          
            \hspace{0.45em}
            \parbox[t]{0.93\linewidth}{
              \raggedright
              \setstretch{2.35} % Increase line spacing
              #2\\[0.7ex]
              \setstretch{1.0} % Restore line spacing
          }
        }

        %% text bubbles
        \tcbset{on line, 
          boxsep=2.2pt, left=4pt,right=4pt,top=4pt,bottom=4pt,
          colframe=lightgray,colback=white, arc=7pt,}
        \newcommand{\keuze}[1]{\textcolor{mainblack}{\tcbox{#1}}}

        \newcommand{\sep}{ \, }
        \newcommand{\seprule}{\noindent\hspace{0cm}\textcolor{lightgray}{\rule{\dimexpr\linewidth-0.1cm\relax}{0.9pt}}\\[0.7ex]}

        \arrayrulecolor{lightgray}

        \node[draw=mainblack, rounded corners=5pt, very thick, inner sep=16pt,
          label={[anchor=south west]north west:{\large \textbf{Enumeraties}}},
          below right=-7.8cm and 3.1cm of fractie, align=left]
        (enumeraties) {
          \begin{tabular}{@{}p{1.0\linewidth} @{\hspace{0.5cm}} | @{\hspace{0.5cm}} p{1.0\linewidth}@{}}
            \makecell[tl]{%
              \enumeration{besluitResultaat}{
                \keuze{Unaniem aangenomen}\sep\keuze{Aangenomen}\sep\keuze{Verworpen}
                \sep\keuze{Geamendeerd aangenomen}\sep\keuze{Onder voorbehoud aangenomen}
              } \\
              \seprule
              \enumeration{dagelijksBestuurType}{
                \keuze{College}\sep\keuze{Gedeputeerde staten}\sep\keuze{Dagelijks bestuur}
              } \\
              \seprule
              \enumeration{fractieStemresultaat}{
                \keuze{Aangenomen}\sep\keuze{Verworpen}\sep\keuze{Verdeeld}
              } \\
              \seprule
              \enumeration{functie}{
                \keuze{Burgemeester}\sep\keuze{Wethouder}\sep\keuze{Raadslid}\sep\keuze{Burgerlid}\sep\keuze{Griffier}
                \sep\keuze{Gedeputeerde}\sep\keuze{Commissaris van de Koning}\sep\keuze{Statenlid}
                \sep\keuze{Provinciesecretaris}\sep\keuze{Secretarisdirecteur}\sep\keuze{Dagelijks bestuurslid}\sep
                \keuze{Algemeen bestuurslid}\sep\keuze{Dijkgraaf}\sep\keuze{Ambtenaar/medewerker}\sep
                \keuze{Adviseur of deskundige}\sep\keuze{Gemeentesecretaris}\sep\keuze{Overig}
              } \\
              \seprule
              \enumeration{geslachtsaanduiding}{
                \keuze{Man}\sep\keuze{Vrouw}\sep\keuze{Anders}\sep\keuze{Onbekend}
              } \\
              \seprule
              \enumeration{keuzeStemming}{
                \keuze{Voor}\sep\keuze{Tegen}\sep\keuze{Afwezig}\sep\keuze{Onthouden}
              }
            } &
            \makecell[tl]{%
            \enumeration{mediabronType}{
              \keuze{Video}\sep\keuze{Audio}\sep\keuze{Transcriptie}
            } \\
            \seprule
            \enumeration{resultaatMondelingeStemming}{
              \keuze{Voor}\sep\keuze{Tegen}\sep\keuze{Gelijk}
            } \\
            \seprule
            \enumeration{rolnaam}{
              \raggedright
              \keuze{Voorzitter}\sep\keuze{Vice-voorzitter}\sep\keuze{Raadslid}\sep\keuze{Statenlid}\sep
              \keuze{Dagelijks bestuurslid}\sep\keuze{Algemeen bestuurslid}\sep\keuze{Inspreker}\sep
              \keuze{Portefeuillehouder}\sep\keuze{Griffier}\sep\keuze{Overig}
            } \\
            \seprule
            \enumeration{status}{
              \keuze{Gepland}\sep\keuze{Gehouden}\sep\keuze{Geannuleerd}
            } \\
            \seprule
            \enumeration{stemmingtype}{
              \keuze{Hoofdelijk}\sep\keuze{Regulier}\sep\keuze{Mondeling}
            } \\
            \seprule
            \enumeration{vergaderingstype}{
              \keuze{Raadsvergadering}\sep\keuze{Commissievergadering}\sep\keuze{Statenvergadering}
              \sep\keuze{Algemene bestuursvergadering}\sep\keuze{Presidium}
            } \\
            \seprule
            \enumeration{vergaderstukType}{
              \keuze{Voorstel}\sep\keuze{Mededeling}\sep\keuze{Amendement}\sep\keuze{Besluitsvormingsstuk}
              \sep\keuze{Toezegging}\sep\keuze{Motie}\sep\keuze{Vraag}\sep\keuze{Antwoord}\sep\keuze{Ingekomen stuk}
            }
          }
          \end{tabular}
        };
        

\end{tikzpicture}

\end{document}
