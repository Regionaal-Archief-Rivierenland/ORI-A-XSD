\documentclass[margin={0pt 15pt 15pt 15pt}]{standalone}

\usepackage[dutch]{babel}

%% font configuration
\usepackage[T1]{fontenc}
\usepackage[default, type1]{sourcesanspro}
\usepackage{inconsolata}
\usepackage[kerning=true, tracking=true, activate={true,nocompatibility}]{microtype} %% tracking is pdftex only

%% symbols
\usepackage{amsmath}
\usepackage{mathtools}
\usepackage{mdsymbol}

%% Tables
\usepackage{makecell}
\usepackage{colortbl} % To enable \arrayrulecolor

%% linestretching
\usepackage{setspace}

%% rounded boxes
\usepackage[most]{tcolorbox}


%% Color definitions
\usepackage[dvipsnames, table]{xcolor}
\definecolor{githubgrey}{HTML}{EAEEF2}
\definecolor{githubdarkblue}{HTML}{053a69}
\definecolor{mainblack}{HTML}{333333}
\definecolor{blueishblack}{HTML}{3A3A3A}
\definecolor{lightgray}{HTML}{D9DDE1}
\definecolor{moderngray}{HTML}{D8D8DB}
\definecolor{darkgreen}{HTML}{43674c}
\definecolor{enumeratiecolor}{HTML}{3C5770}

%% tikz
\usepackage{tikz}
\usetikzlibrary{arrows}
\usetikzlibrary{fit}
\usetikzlibrary{calc}
\usetikzlibrary{arrows.meta}
\usetikzlibrary{shapes.multipart}
\usetikzlibrary{shapes.misc}
\usetikzlibrary{positioning}
%% needed for making svgs with white backgrounds
\usetikzlibrary{backgrounds}

\begin{document}

%% set default font to sans serif
\sffamily


%% TODO
%% add arrow from vergadering to MDTO
%% - be consistent in ur use of baselineskips (and amounts)
%% - moeten verplichte relaties ook dik gedrukt/anders getypeset
%% - maybe bullets instead of plusses? Making a proper list also gives more flexibility
%%    - all the more because the plusses serve little purposes

\newcommand{\kyshiftbeg}{0.25cm}
\newcommand{\kyshiftend}{0.70cm}
\newcommand{\kxshiftbeg}{0.35cm}
\newcommand{\kxshiftend}{0.80cm}
\newcommand\mystrut{\rule[-2.5pt]{0pt}{12pt}}
%% datatype typesetting
\newcommand{\datatype}[2][text]{%
    \tcbox[
      on line,
      boxsep=0pt,
      left=2.5pt,
      right=2.5pt,
      top=0pt,
      bottom=0pt,
      enlarge top initially by=-14pt,
      enlarge bottom by=-14pt,
      enlarge right by=-4pt,
      enlarge left by=-2pt,
      opacityframe=0,
      colback=githubgrey,
      fontupper={\ttfamily\mystrut},
      fontlower={\ttfamily\mystrut}]
      {\textcolor{blueishblack}{\textbf{#2}}}
}

\newcommand{\umltype}[1]{
  {\textbf{\textcolor{gray}{\guillemotleft#1\guillemotright}}}
}

\newcommand{\enumeratietype}[1]{enumeratie \textcolor{darkgreen}{#1}}

%% Header of boxes
\newcommand{\objectheader}[2]{
  \umltype{#1} \\
  {\large \textcolor{mainblack}{{\fontseries{b}\selectfont #2}}}
  \nodepart[align=left]{two}
}

%% TODO: try if a regular box looks better (maybe make the whole box a certain color? or just the bottom half)
%% Attribute typesetting
\newcommand{\attrib}[3]{\textcolor{mainblack}{+ \; #1:}\, \datatype{#2} \; {
    %% \figureversion{text} %% doesn't work with every font?
    
    \textcolor{gray}{\oldstylenums{\scshape#3}}}}


\begin{tikzpicture}[
    %% needed to fix svg
    background rectangle/.style={fill=white}, show background rectangle,
    %% not sure why the "header" is still centered after align=left, but oh well
    %% changerectangle split part fill for colors
    ORI object/.style={rectangle split, rectangle split parts=2,
      rectangle split part fill={white,white}, draw=mainblack, rounded corners=5pt, fill=white,
      very thick, inner sep=10pt,
    },
    ORI relatieklasse/.style={rectangle split, rectangle split parts=2,
      rectangle split part fill={white,white}, rounded corners=5pt,
      very thick, inner sep=10pt, draw=gray!86, fill=white,
      },
    ORI referentie/.style={draw=mainblack, rounded corners=5pt, align=center, very thick, inner sep=10pt, draw=gray!86},
    relatiepijl/.style={-{Straight Barb[length=4mm,width=6mm]}, very thick, draw=mainblack, text=mainblack},
    gegevenspijl/.style={-{Straight Barb[length=4mm,width=6mm]}, very thick, dashed, draw=mainblack},
    generalisatiepijl/.style={-{Latex[round,open,length=4mm,width=6mm]}, very thick, draw=mainblack},
    dashedpijl/.style={very thick, dashed, text=mainblack, draw=gray!86},
    kardinaliteit/.style={text=mainblack}, 
    align=center]
  
  \node (agendapunt) [ORI object, align=center]
        {
          \objectheader{Objecttype}{Agendapunt}

          \attrib{\textbf{ID}}{string}{} \\
          \attrib{Omschrijving}{string}{[0..1]} \\
          \attrib{Gepland volgnummer}{string}{[0..1]} \\
          \attrib{Volgnummer}{string}{[0..1]} \\
          \attrib{Geplande starttijd}{datum + tijd}{[0..1]} \\
          \attrib{Geplande eindtijd}{datum + tijd}{[0..1]} \\
          \attrib{Starttijd}{datum + tijd}{[0..1]} \\
          \attrib{Eindtijd}{datum + tijd}{[0..1]} \\
          \attrib{Kenmerk}{string}{[0..1]} \\
          \attrib{\textbf{Titel}}{string}{} \\
          \attrib{Hamerstuk}{boolean}{[0..1]} \\
          \attrib{Behandeld}{boolean}{[0..1]} \\
          \attrib{Besloten}{boolean}{[0..1]} \\
          \attrib{Bestuurslaag}{bestuurslaagGegevens}{[0..1]}
        };


   \node (vergadering) [ORI object, align=center, left=17cm of agendapunt]
        {
          \objectheader{Objecttype}{Vergadering}

          \attrib{\textbf{ID}}{string}{} \\
          \attrib{\textbf{Naam}}{string}{} \\
          \attrib{Vergadertoelichting}{string}{[0..1]} \\
          \attrib{Georganiseerd door gremium}{gremiumGegevens}{[0..1]} \\
          \attrib{Vergaderingstype}{\enumeratietype{vergaderingstype}}{[0..1]} \\
          \attrib{Locatie}{string}{[0..1]} \\
          \attrib{Status}{\enumeratietype{status}}{[0..1]} \\
          \attrib{Geplande vergaderdatum}{datum}{[0..1]} \\
          \attrib{\textbf{Vergaderdatum}}{datum}{} \\
          \attrib{Geplande aanvang vergadering}{datum + tijd}{[0..1]} \\
          \attrib{Gepland einde vergadering}{datum + tijd}{[0..1]} \\
          \attrib{Aanvang vergadering}{datum + tijd}{[0..1]} \\
          \attrib{Einde vergadering}{datum + tijd}{[0..1]} \\
          \attrib{Publicatiedatum}{datum + tijd}{[0..1]} \\
          \attrib{Bestuurslaag}{bestuurslaagGegevens}{[0..1]}
        };

   \node (deelnemer) [ORI object, align=center, below=3.3cm of vergadering]
        {
          \objectheader{Objecttype}{Aanwezige deelnemer}

          \attrib{ID}{string}{[0..1]} \\
          \attrib{Rolnaam}{\enumeratietype{rolnaam}}{[0..1]} \\
          \attrib{Organisatie}{string}{[0..1]} \\
          \attrib{Deelnemerspositie}{string}{[0..1]} \\
          \attrib{Aanvang aanwezigheid}{datum + tijd}{[0..1]} \\
          \attrib{Einde aanwezigheid}{datum + tijd}{[0..1]}
        };

    \node (stemming) [ORI object, below = 3.9cm of agendapunt, align=center]
        {
          \objectheader{Objecttype}{Stemming}

          \attrib{\textbf{ID}}{string}{} \\
          \attrib{Stemmingtype}{\enumeratietype{stemmingtype}}{[0..1]}\\
          \attrib{Resultaat mondelinge stemming}{\enumeratietype{resultaatMondelingeStemming}}{[0..1]} \\
          \attrib{Toezegging}{string}{[0..1]} \\
          \attrib{Resultaat stemming over personen}{string}{[0..1]} \\
          \attrib{Stemming over personen}{stemmingOverPersonenGegevens}{[0..*]}
        };

    \node (fractie) [ORI object, below = 8.5cm of stemming, align=center]
        {
          \objectheader{Objecttype}{Fractie}

          \attrib{\textbf{ID}}{string}{} \\
          \attrib{\textbf{Fractienaam}}{string}{}\\
          \attrib{Bestuurslaag}{bestuurslaagGegevens}{[0..1]}
        };
        
   \node (stem) [ORI relatieklasse, align=center, at = ($(deelnemer.east)!0.5!(stemming.west)$), yshift=2.8cm]
        {
          \objectheader{Relatieklasse}{Stem}

          \attrib{ID}{string}{[0..1]} \\
          \attrib{\textbf{Keuze stemming}}{\enumeratietype{keuzeStemming}}{}
        };
   \node (stemresultaatperfractie) [ORI relatieklasse, align=center, below right = 6.5cm and -4.5cm of stem]
        {
          \objectheader{Relatieklasse}{Stemresultaat per fractie}

          \attrib{ID}{string}{[0..1]} \\
          \attrib{Fractie stemresultaat}{\enumeratietype{fractieStemresultaat}}{[0..1]}
        };
   \node (dagelijksbestuur) [ORI object, align=center, below = 5.2cm of fractie]
        {
          \objectheader{Objecttype}{Dagelijks bestuur}

          \attrib{\textbf{ID}}{string}{} \\
          \attrib{\textbf{Dagelijks bestuur naam}}{string}{} \\
          \attrib{Dagelijks bestuur type}{\enumeratietype{dagelijksBestuurType}}{[0..1]} \\
          \attrib{\textbf{Bestuurslaag}}{bestuurslaagGegevens}{}
        };

   \node (spreekfragment) [ORI relatieklasse, above = 17.5cm of stem, xshift=-1.6cm]
        {
          \objectheader{Relatieklasse}{Spreekfragment}

          \attrib{ID}{string}{[0..1]} \\
          \attrib{Aanvang spreekfragment}{datum + tijd}{[0..1]} \\
          \attrib{Einde spreekfragment}{datum + tijd}{[0..1]} \\
          \attrib{Taal}{taal}{[0..1]} \\
          \attrib{Tekst spreekfragment}{string}{[0..1]} \\
          \attrib{Titel spreekfragment}{string}{[0..1]} \\
          \attrib{Positie notulen}{string}{[0..1]} \\
          \attrib{Audio tijdsaanduiding aanvang}{integer \textcolor{githubdarkblue}{of} tijd}{[0..1]} \\
          \attrib{Audio tijdsaanduiding einde}{integer \textcolor{githubdarkblue}{of} tijd}{[0..1]} \\
          \attrib{Video tijdsaanduiding aanvang}{integer \textcolor{githubdarkblue}{of} tijd}{[0..1]} \\
          \attrib{Video tijdsaanduiding einde}{integer \textcolor{githubdarkblue}{of} tijd}{[0..1]}
        };
        
   \node (natuurlijkpersoon) [ORI object, align=center, left=5.0cm of deelnemer]
        {
          \objectheader{Objecttype}{Natuurlijk persoon}

          \attrib{\textbf{ID}}{string}{} \\
          \attrib{Geslachtsaanduiding}{\enumeratietype{geslachtsaanduiding}}{[0..1]} \\
          \attrib{Functie}{\enumeratietype{functie}}{[0..1]} \\
          \attrib{\textbf{Naam}}{naamGegevens}{} \\
          \attrib{Nevenfunctie}{nevenfunctieGegevens}{[0..*]}
        };

   \node (naam) [ORI object, below = 2.2cm of natuurlijkpersoon, align=center, draw=gray!86, xshift=-3cm]
        {
          \objectheader{Gegevensgroeptype}{Naam}

          \attrib{\textbf{Achternaam}}{string}{} \\
          \attrib{Tussenvoegsel}{string}{[0..1]} \\
          \attrib{Voorletters}{string}{[0..1]} \\
          \attrib{Voornamen}{string}{[0..1]} \\
          \attrib{Volledige naam}{string}{[0..1]}
        };

   \node (nevenfunctie) [ORI object, above left = 2.2cm and -9cm of natuurlijkpersoon, align=center, draw=gray!86]
        {
          \objectheader{Gegevensgroeptype}{Nevenfunctie}

          \attrib{\textbf{Omschrijving nevenfunctie}}{string}{} \\
          \attrib{Naam organisatie nevenfunctie}{string}{[0..1]} \\
          \attrib{Aantal uren per maand}{integer}{[0..1]} \\
          \attrib{bezoldigd}{boolean}{[0..1]} \\
          \attrib{nevenfunctie vanwege lidmaatschap}{boolean}{[0..1]} \\
          \attrib{Datum melding}{datum}{[0..1]} \\
          \attrib{Datum aanvang nevenfunctie}{datum}{[0..1]} \\
          \attrib{Datum einde nevenfunctie}{datum}{[0..1]}
        };

   \node (fractielidmaatschap) [ORI relatieklasse, below = 4.0cm of deelnemer, align=center]
        {
          \objectheader{Relatieklasse}{Fractielidmaatschap}

          \attrib{ID}{string}{[0..1]} \\
          \attrib{Datum begin fractielidmaatschap}{datum}{[0..1]} \\
          \attrib{Datum einde fractielidmaatschap}{datum}{[0..1]} \\
          \attrib{voorzitter}{boolean}{[0..1]}
        };

   \node (dagelijksbestuurlidmaatschap) [ORI relatieklasse, align=center, below = 6.45cm of fractielidmaatschap]
        {
          \objectheader{Relatieklasse}{Dagelijks bestuur lidmaatschap}

          \attrib{ID}{string}{[0..1]} \\
          \attrib{Datum begin dagelijks bestuur lidmaatschap}{datum}{[0..1]} \\
          \attrib{Datum eind dagelijks bestuur lidmaatschap}{datum}{[0..1]}
        };

  \node (besluit) [ORI object, above right = 0.5cm and 0.9cm of stemming, align=center]
        {
          \objectheader{Objecttype}{Besluit}

          \attrib{ID}{string}{[0..1]} \\
          \attrib{Besluit toelichting}{string}{[0..1]} \\
          \attrib{Toezegging}{string}{[0..1]} \\
          \attrib{\textbf{Besluit resultaat}}{\enumeratietype{besluitResultaat}}{}
        };

  \node (stemmingOverPersonenGegevens) [ORI object, below right = 5.3cm and -9.2cm of besluit, align=center, draw=gray!86]
        {
          \objectheader{Gegevensgroeptype}{Stemming over personen}

          \attrib{\textbf{Naam kandidaat}}{string}{} \\
          \attrib{Aantal uitgebrachte stemmen}{integer}{[0..1]}
        };

   \node (informatieobject) [ORI relatieklasse, above right = -4.6cm and 7.2cm of agendapunt, align=center]
        {
          \objectheader{Relatieklasse}{Informatieobject}

          \attrib{Informatieobject type}{begripGegevens}{[0..1]}
        };

  %% inner MDTO rectangle
  \node (mdto-inner) [rectangle, draw=mainblack, text=mainblack, ultra thick, right = 3.2cm of informatieobject, inner sep=40pt, fill=githubgrey, align=center]
        {
          {\LARGE \fontseries{sb}\selectfont MDTO\,/\,TMLO}
        };

  % Outer MDTO rectangle (this is what arrows should connect to)
  \node (mdto) [fit=(mdto-inner), draw=mainblack, ultra thick, inner sep=1.1cm] {};

  %% currently unused nodes:
  
  %% \node (bestuurslaag) [ORI referentie, above=20cm of natuurlijkpersoon] {
  %%   \umltype{Gegevensgroeptype}\\
  %%   \large \textcolor{mainblack}{{\fontseries{b}\selectfont Bestuurslaag}}
    
  %% };

  %% \node (waterschap) [ORI object, below=3cm of bestuurslaag, xshift=-7cm, draw=gray!86] {
  %%   \objectheader{Gegevensgroeptype}{Waterschap}
  %%   \attrib{\textbf{Waterschapcode}}{string}{} \\
  %%   \attrib{Waterschapnaam}{string}{[0..1]}
  %% };

  %% \node (gemeente) [ORI object, below=3cm of bestuurslaag, xshift=0, draw=gray!86] {
  %%   \objectheader{Gegevensgroeptype}{Gemeente}
  %%   \attrib{\textbf{Gemeentecode}}{string}{} \\
  %%   \attrib{Gemeentenaam}{string}{[0..1]}
  %% };

  %% \node (provincie) [ORI object, below=3cm of bestuurslaag, xshift=7cm, draw=gray!86] {
  %%   \objectheader{Gegevensgroeptype}{Provincie}
  %%   \attrib{\textbf{Provinciecode}}{string}{} \\
  %%   \attrib{Provincienaam}{string}{[0..1]}
  %% };

  %% arrows

  %% \draw[generalisatiepijl] (waterschap) |-
  %% node [pos=0.2, fill=white] {\umltype{Generalisatie}}
  %% (bestuurslaag);
  %% \draw[generalisatiepijl] (provincie) |-
  %% node [pos=0.2, fill=white] {\umltype{Generalisatie}}
  %% (bestuurslaag);
  %% \draw[generalisatiepijl] (gemeente) --
  %% node [pos=0.5, fill=white] {\umltype{Generalisatie}}
  %% (bestuurslaag);
  
        %% FIXME: cardinalities won't be alligned, because pos= is relative to arrow length
        %% stemming -> agendapunt
  \draw [relatiepijl] (stemming) --
  node [anchor=center, fill=white] {heeft betrekking op\\\umltype{Relatiesoort}}
  node [kardinaliteit, pos=0.00, yshift=\kyshiftbeg, right]  {\oldstylenums{1..*}}
  node [kardinaliteit, pos=1.0, yshift=-\kyshiftend, right] {\oldstylenums{1}}
  (agendapunt);
  %% stemming -> besluit
  \draw[relatiepijl] ($(stemming.north)+(2.5,0)$) |- (besluit.west)
  node [pos=0.74, above] {leidt tot}
  node [pos=0.74, below] {\umltype{Relatiesoort}}
  node [kardinaliteit, pos=0.0, yshift=\kyshiftbeg, right] {\oldstylenums{0..*}}
  node [kardinaliteit, above, pos=1.0, xshift=-\kxshiftend] {\oldstylenums{0..1}};
  %% stemming -> stemming over personen
  \draw[gegevenspijl] ($(stemming.south) + (2.5, -0.0cm)$) |- (stemmingOverPersonenGegevens.west);
  %% stemming -> MDTO
  \draw[relatiepijl] ($(stemming.east) + (0, 0.0cm)$) -|
  node [kardinaliteit, pos=0, above, xshift=\kxshiftbeg]  {\oldstylenums{0..*}}
  node [kardinaliteit, pos=1, right, yshift=-\kyshiftend]  {\oldstylenums{0..*}}
  node [pos=0.302, above] {heeft betrekking op}
  node [pos=0.302, below] {\umltype{Relatiesoort}}
  coordinate[pos=0.302] (tempcoordinate) % coordinate along the line
  (mdto.south);
  %% draw dashed line to informatieobject
  \coordinate (lineendnorth) at (tempcoordinate|-informatieobject.south);
  \draw[dashedpijl] ($(tempcoordinate) + (0, 0.5cm)$) -- (lineendnorth);

  %% deelnemer -> vergadering
  \draw[relatiepijl] (deelnemer)  --
  node [kardinaliteit, pos=0.0, yshift=\kyshiftbeg, right] {\oldstylenums{0..*}}
  node [kardinaliteit, pos=1.0, yshift=-\kyshiftend, right] {\oldstylenums{1}}
  node [anchor=center, fill=white] {neemt deel aan\\\umltype{Relatiesoort}} (vergadering);
  %% deelnemer -> persoon
  \draw[relatiepijl] (deelnemer) --
  node [kardinaliteit, pos=0.0, xshift=-\kxshiftbeg, above] {\oldstylenums{0..*}}
  node [kardinaliteit, pos=1.0, xshift=0.6cm, above] {\oldstylenums{1}}
  node [above] {is natuurlijk persoon}
  node [below] {\umltype{Relatiesoort}}
  (natuurlijkpersoon);
  %% deelnemer -> stemming
  \draw[dashedpijl] ($(deelnemer.east)!0.5!(stemming.west) + (0, -.19)$) -- (stem);
  \coordinate (deelnemershifted) at ($(deelnemer.east) - (0,0.5cm)$);
  \coordinate (stemmingshifted) at ($(stemming.west) - (0,0.5cm)$);
  % Use the `|-` operator to align vertically while shifting horizontally
  \draw[relatiepijl] 
  ($(deelnemershifted |- stemmingshifted)$) --
  node [kardinaliteit, pos=0.0, xshift=\kxshiftbeg, above] {\oldstylenums{0..*}}
  node [kardinaliteit, pos=1.0, xshift=-\kxshiftend, above] {\oldstylenums{0..*}}
  node [above, fill=white] {neemt deel aan} 
  node [below] {\umltype{Relatiesoort}} 
  (stemmingshifted.west);
  %% deelnemer -> agendapunt (by way of spreekfragment)
  \draw[relatiepijl] ($(deelnemer.north) + (3cm, 0)$) --
  node [kardinaliteit, pos=0.0, yshift=\kyshiftbeg, right] {\oldstylenums{0..*}}
  ($(deelnemer.north) + (3cm, 2.6cm)$) --
  node [above, pos=0.75] {spreekt tijdens}
  node [below, pos=0.75] {\umltype{Relatiesoort}}
  coordinate[pos=0.75] (tempcoordinate) %% define relatie line starting position
  ($(deelnemer.north) + (12cm, 2.6cm)$) |-
  node [kardinaliteit, pos=1.0, xshift=-\kxshiftend, above] {\oldstylenums{0..*}}
  ($(agendapunt.west) -  (0, 3.0cm)$);        
  % Draw the dashed line
  \draw[dashedpijl] ($(tempcoordinate) + (0, 0.65cm)$) -- 
  (tempcoordinate |- spreekfragment.south);

  
  %% agendapunt -> vergadering
  \draw[relatiepijl] (agendapunt) -- (vergadering)
  node [pos=0.5, above] {wordt behandeld tijdens}
  node [pos=0.5, below] {\umltype{Relatiesoort}}
  node [kardinaliteit, pos=0.00, xshift=-\kxshiftbeg,  above] {\oldstylenums{0..*}}
  node [kardinaliteit, pos=0.965, above] {\oldstylenums{1}};
  %% agendapunt -> agendapunt
  \draw[relatiepijl] ($(agendapunt.east)+(0, 2.8)$) |-
  node [kardinaliteit, above, xshift = \kxshiftbeg] {\oldstylenums{0..1}} 
  ($(agendapunt.east)+(3.0, 2.8)$) |- ($(agendapunt.east)+(3.0, -0.8)$)
  node [fill=white, pos=0.25] {heeft als subagendapunt\\\umltype{Relatiesoort}} --
  ($(agendapunt.east)+(0, -0.8)$)
  node [kardinaliteit, above, xshift = 0.8cm] {\oldstylenums{0..*}};
  %% agendapunt -> mdto
  \draw[relatiepijl] 
  ($(agendapunt.east) - (0, 2.5cm)$) --
  node [kardinaliteit, pos=0, above, xshift=\kxshiftbeg] {\oldstylenums{0..*}}
  ++(2cm,0) -| ($(mdto.south) - (2cm, 0cm)$)
  coordinate[pos=0.25] (tempcoordinate)
  node [kardinaliteit, pos=1, right, yshift=-\kyshiftend] {\oldstylenums{0..*}}
  node [pos=0.25, above] {heeft als bijlage}
  node [pos=0.25, below] {\umltype{Relatiesoort}};
  % Draw the dashed line
  \coordinate (lineendnorth) at (tempcoordinate|-informatieobject.south);
  \draw[dashedpijl] ($(tempcoordinate) + (0, 0.52cm)$) -- (lineendnorth);
  %% agendapunt -> natuurlijkpersoon
  %% halfwaypoint stem and stemming
  \coordinate (midpoint) at ($(stem.east)!.5!(stemming.west) + (-0.5cm,0)$);
  % Project the midpoint to the Y-coordinate of stem.north
  \coordinate (midpointadjusted) at (midpoint|-stem.north);
  \draw[very thick, draw=mainblack] ($(agendapunt.south) - (3.0cm,0)$) |-
  node [kardinaliteit, pos=0, right, yshift=-\kyshiftbeg] {\oldstylenums{0..*}}
  (midpointadjusted) --
  ($(midpointadjusted) - (0cm, 4.77cm)$)
  % save bump location
  coordinate[pos=1.122] (bumpstart); % Save the starting point for the bump
  % Create the bump
  \draw[very thick, draw=mainblack] (bumpstart)
  arc[start angle=270, end angle=90, radius=0.3cm];
  \draw[relatiepijl, yshift=-1.2cm] ($(bumpstart) + (0,0.02cm)$) --
  ($(bumpstart) - (0,3.3cm)$) -|
  node [pos=0.17, above] {heeft behandelend ambtenaar}
  node [pos=0.17, below] {\umltype{Relatiesoort}}
  node [kardinaliteit, pos=1, right, yshift=-\kyshiftbeg, xshift=\kxshiftbeg] {\oldstylenums{0..*}}
  ($(natuurlijkpersoon.south) + (3.9cm, 0)$);
  
  %% fractie -> stemming
  \draw[relatiepijl] (fractie) --
  coordinate[pos=0.5] (tempcoordinate) %% define tmp halfway coord (for allignment reasons)
  node [kardinaliteit, pos=0.00, yshift=0.25cm, right] {\oldstylenums{0..*}}
  node [kardinaliteit, pos=1.0, yshift=-0.75cm, right] {\oldstylenums{0..*}}
  (stemming);
  \coordinate (linestartfractiestemresultaat) at (tempcoordinate |- stemresultaatperfractie.east);
  %% dashed line to stemresultaat per fractie
  \draw[dashedpijl] ($(linestartfractiestemresultaat)-(0,0.3cm)$) --
  node [pos=0, anchor=center, fill=white] {neemt deel aan\\\umltype{Relatiesoort}}
  ($(stemresultaatperfractie.east)-(0,0.3cm)$);
  
  
  %% One could argue that this should just be another attribute
  %% vergadering -> vergadering
  \draw[relatiepijl] ($(vergadering.west)+(0,2.8)$) |-
  node [kardinaliteit, above, xshift = -0.4cm] {\oldstylenums{0..1}}
  ($(vergadering.west)+(-3.0,2.8)$) |- ($(vergadering.west)+(-3.0,-0.8)$)
  node [fill=white, pos=0.25] {heeft als deelvergadering\\\umltype{Relatiesoort}} --
  ($(vergadering.west)+(0,-0.8)$)
  node [kardinaliteit, above, xshift = -0.8cm] {\oldstylenums{0..*}} ;

  %% vergadering -> mdto (bijlage)
  \draw[relatiepijl] ($(vergadering.east)+(0, 3)$) -|
  node [kardinaliteit, above, pos=0, xshift=\kxshiftbeg] {\oldstylenums{0..*}}
  ($(vergadering.east)+(11, 5.5)$) -| 
  node [kardinaliteit, right, pos=1, yshift=\kyshiftend] {\oldstylenums{0..*}}
  node [anchor=center, above, pos=0.335] {heeft als bijlage}
  node [anchor=center, below, pos=0.335] {\umltype{Relatiesoort}}
  coordinate[pos=0.335] (tempcoordinate) %% define tmp halfway coord
  ($(mdto.north)+(-2.8cm,0)$);

  % Draw the dashed line
  \coordinate (lineendsouth) at (tempcoordinate|-informatieobject.north);
  \draw[dashedpijl] ($(tempcoordinate) + (0, -0.5cm)$) -- (lineendsouth);
  
  %% vergadering -> mdto (notulen)
  \draw[relatiepijl] ($(vergadering.north)+(4, 0)$) --
  node [kardinaliteit, right, pos=0, yshift=\kyshiftbeg] {\oldstylenums{0..1}}
  ($(vergadering.north)+(4, 2.5)$) -| 
  node [kardinaliteit, right, pos=1, yshift=\kyshiftend] {\oldstylenums{0..1}}
  node [anchor=center, above, pos=0.32] {is genotuleerd in}
  node [anchor=center, below, pos=0.32] {\umltype{Relatiesoort}}
  ($(mdto.north)+(-1cm,0)$);
  

  %% persoon -> naam
  \draw[gegevenspijl] ([xshift=-3cm] natuurlijkpersoon.south) -- ([xshift=-3cm] naam);
  %% persoon -> nevenfunctie
  \draw[gegevenspijl] ([xshift=-3cm] natuurlijkpersoon.north) -- ([xshift=-3cm] natuurlijkpersoon.north |- nevenfunctie.south);
  %% persoon -> fractie (by way of fractielidmaatschap)
  % calculate endpoint
  \coordinate (Y) at ($(fractielidmaatschap.south)-(0.0cm,2.5cm)$);
  \coordinate (endpoint) at (fractie.west|-Y);
  \draw[relatiepijl] (natuurlijkpersoon) ($(natuurlijkpersoon.south)+(2.4,0)$)  |-
  node[kardinaliteit, pos=0, right, yshift=-\kyshiftbeg] {\oldstylenums{0..*}}
  coordinate[pos=1.0] (linestartfractielidmaatschap) % coordinate along the line
  ($(fractielidmaatschap.south)-(0.0cm,2.5cm)$) --
  node[kardinaliteit, pos=1, above, xshift=-\kxshiftend] {\oldstylenums{0..*}}
  (endpoint.west);
  %% dashed line
  \draw[dashedpijl, yshift=-3cm] (linestartfractielidmaatschap) --
  node [above, fill=white, pos=0.0] {is lid van}
  node [below, pos=0.0] {\umltype{Relatiesoort}}
  (fractielidmaatschap.south);
  %% persoon -> dagelijks bestuur (by way of dagelijksbestuurlidmaatschap)
  \draw[relatiepijl] (natuurlijkpersoon) ($(natuurlijkpersoon.south)+(1.0,0)$) |-
  node[kardinaliteit, pos=0, right, yshift=-\kyshiftbeg] {0..*}
  node[kardinaliteit, pos=1, above, xshift=-\kxshiftend] {0..*}
  ($(dagelijksbestuur.west) - (0, 0.5cm)$);
  %% dashed line
  \draw[dashedpijl] (dagelijksbestuurlidmaatschap.south) --
  node [above, fill=white, pos=1.05] {is lid van}
  node [below, pos=1.05] {\umltype{Relatiesoort}}
  ($(dagelijksbestuurlidmaatschap.south) - (0, 1.8cm)$);

  %% enumeraties

  \newcommand{\enumeration}[2]{
    \umltype{Enumeratie} \; \textcolor{mainblack}{{\fontseries{sb}\selectfont #1}}\\[0.9ex]
    
    \hspace{0.45em}
    \parbox[t]{0.93\linewidth}{
      \raggedright
      \setstretch{2.35} % Increase line spacing
      #2\\[0.7ex]
      \setstretch{1.0} % Restore line spacing
    }
  }

  %% text bubbles
  \tcbset{on line, 
    boxsep=2.2pt, left=4pt,right=4pt,top=4pt,bottom=4pt,
    colframe=moderngray,colback=white, arc=7pt,}
  \newcommand{\keuze}[1]{\tcbox{\textcolor{mainblack}{#1}}}

  \newcommand{\sep}{ \, }
  \newcommand{\seprule}{\noindent\hspace{0cm}\textcolor{moderngray}{\rule{\dimexpr\linewidth-0.1cm\relax}{1.0pt}}\\[0.7ex]}

  \arrayrulecolor{moderngray}
  \setlength{\arrayrulewidth}{0.7pt} %% a slightly less thick line looks bettter here imo

  \node[draw=mainblack, rounded corners=5pt, very thick, inner sep=16pt,
    label={[anchor=south west]north west:{\large \textcolor{mainblack}{\fontseries{b}\selectfont Enumeraties}}},
    below right=-7.8cm and 3.1cm of fractie, align=left]
  (enumeraties) {
    \begin{tabular}{@{}p{1.0\linewidth} @{\hspace{0.5cm}} | @{\hspace{0.5cm}} p{1.0\linewidth}@{}}
      \makecell[tl]{%
        \enumeration{besluitResultaat}{
          \keuze{Unaniem aangenomen}\sep\keuze{Aangenomen}\sep\keuze{Verworpen}
          \sep\keuze{Geamendeerd aangenomen}\sep\keuze{Onder voorbehoud aangenomen}
        } \\
        \seprule
        \enumeration{dagelijksBestuurType}{
          \keuze{College}\sep\keuze{Gedeputeerde staten}\sep\keuze{Dagelijks bestuur}
        } \\
        \seprule
        \enumeration{fractieStemresultaat}{
          \keuze{Aangenomen}\sep\keuze{Verworpen}\sep\keuze{Verdeeld}
        } \\
        \seprule
        \enumeration{functie}{
          \keuze{Burgemeester}\sep\keuze{Wethouder}\sep\keuze{Raadslid}\sep\keuze{Burgerlid}\sep\keuze{Griffier}
          \sep\keuze{Gedeputeerde}\sep\keuze{Commissaris van de Koning}\sep\keuze{Statenlid}
          \sep\keuze{Provinciesecretaris}\sep\keuze{Secretarisdirecteur}\sep\keuze{Dagelijks bestuurslid}\sep
          \keuze{Algemeen bestuurslid}\sep\keuze{Dijkgraaf}\sep\keuze{Ambtenaar/medewerker}\sep
          \keuze{Adviseur of deskundige}\sep\keuze{Gemeentesecretaris}\sep\keuze{Overig}
        } \\
        \seprule
        \enumeration{geslachtsaanduiding}{
          \keuze{Man}\sep\keuze{Vrouw}\sep\keuze{Anders}\sep\keuze{Onbekend}
        } \\
        \seprule
        \enumeration{keuzeStemming}{
          \keuze{Voor}\sep\keuze{Tegen}\sep\keuze{Afwezig}\sep\keuze{Onthouden}
        }
      } &
      \makecell[tl]{%
        \enumeration{resultaatMondelingeStemming}{
          \keuze{Voor}\sep\keuze{Tegen}\sep\keuze{Gelijk}
        } \\
        \seprule
        \enumeration{rolnaam}{
          \raggedright
          \keuze{Voorzitter}\sep\keuze{Vice-voorzitter}\sep\keuze{Raadslid}\sep\keuze{Statenlid}\sep
          \keuze{Dagelijks bestuurslid}\sep\keuze{Algemeen bestuurslid}\sep\keuze{Inspreker}\sep
          \keuze{Portefeuillehouder}\sep\keuze{Griffier}\sep\keuze{Overig}
        } \\
        \seprule
        \enumeration{status}{
          \keuze{Gepland}\sep\keuze{Gehouden}\sep\keuze{Geannuleerd}
        } \\
        \seprule
        \enumeration{stemmingtype}{
          \keuze{Hoofdelijk}\sep\keuze{Regulier}\sep\keuze{Mondeling}
        } \\
        \seprule
        \enumeration{vergaderingstype}{
          \keuze{Raadsvergadering}\sep\keuze{Commissievergadering}\sep\keuze{Statenvergadering}
          \sep\keuze{Algemene bestuursvergadering}\sep\keuze{Presidium}
        } \\
        \seprule
        \enumeration{informatieobjectType}{
          \keuze{Voorstel}\sep\keuze{Mededeling}\sep\keuze{Amendement}\sep\keuze{Besluitsvormingsstuk}
          \sep\keuze{Toezegging}\sep\keuze{Motie}\sep\keuze{Vraag}\sep\keuze{Antwoord}\sep\keuze{Ingekomen stuk}
        }
      }
    \end{tabular}
  };
  

\end{tikzpicture}

\end{document}
